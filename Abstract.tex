\documentclass[]{scrartcl}

\newcommand\tab[1][0.7cm]{\hspace*{#1}}
\usepackage{amssymb}
\usepackage{amsmath}

\usepackage{lipsum}
\usepackage{setspace}
\renewcommand{\baselinestretch}{1.5} 

\date{}

%opening
\title{Platform\u{a} Web}
\author{Iovu Gabriel}

\begin{document}

\maketitle



\tab Aceasta lucrare ofera o solutie pentru retelele de Organizatii Non-Guvernamentale, atat romane cat si straine, care beneficiaza de finantare de la Comisia Europeana pentru proiecte de tipul KA1, KA2 si KA3. In momentul de fata, ONG-urile folosesc grupuri publice de Facebook in incercarea de a promova proiectele, problemele fiind evidente: numarul mare de grupuri existente, un mod de comunicare ingreunat intre ONG si participant, transparenta procesului de selectie etc. \\
\tab Scopul primar al acestei lucrari este de a dezvolta o platforma web care va facilita selectia membrilor pentru proiectele mentionate, diseminarea rezultatelor, crearea de parteneriate si la cresterea vizibilitatii programului Erasmus+. \\
\tab Al doilea scop al acestei lucrari este de a un mediu de selectie automata a participantilor in proiecte, bazandu-se pe preferintelor participantilor asupra proiectelor disponibile. Pentru a realiza aceasta selectie automata a participantilor am folosind un algoritm de tip Stabe-match, in care entitatile implicate sunt „participantii”, „ONG-urile” si „proiectele” oferite de aceste ONG-uri. \\
\tab Atat partea de front-end cat si cea de back-end au fost dezvoltate simultan. Pentru partea de back-end, tehnologia primara pe care am utilizat-o este Python, folosind framework-ul Flask. Pentru partea de front-end am creat ofera o interfata „user-firendly”, pe baza framework-ului Bootstrap. Serveul ofera un mediu sigur in care autentificarea si autorizarea sunt facute, iar datele sunt stocate in baze de date MySQL.


\end{document}
